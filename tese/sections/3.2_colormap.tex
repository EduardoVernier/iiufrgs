%!TEX root = ../vernier.tex
\section{Colormap} \label{sec:colormap}
Colormaps are mapping functions that for every point of the domain of interest, assign to it a color based on the scalar value at that point. In this tool, three different color mapping schemes based on the ColorBrewer's \cite{ref:colorbrewer} samples were implemented. A sequential colormap was used to display the current normalized value of a chosen metric, a diverging colormap was used to show the increase/decrease of a given metric from revision $T_{n-1}$ to $T_{n}$, and a categorical colormap was used to group classes from the same package into a single color.

\begin{figure}[H]
	\centering
	\includegraphics[width=0.6\textwidth]{figures/seq.png}
	\caption{Sequential Colormap}
	\label{fig:seq}
\end{figure}

\begin{figure}[H]
	\centering
	\includegraphics[width=0.6\textwidth]{figures/div.png}
	\caption{Diverging Colormap}
	\label{fig:div}
\end{figure}

\begin{figure}[H]
	\centering
	\includegraphics[width=0.6\textwidth,height=1.0cm]{figures/quali.png}
	\caption{Qualitative Colormap}
	\label{fig:quali}
\end{figure}

Combining colormaps with the hierarchical display techniques, we can start to understand more complex phenomena. On Figure \ref{fig:colormap_hier} we have used the Sequential Colormap to encode the same metric that is represented by the area of treemap rectangles (i.e. number of lines of code) in the last collected revision of the ExoPlayer project. Between the two representations there is a color legend that displays the full colormap and presents the minimum and maximum values of the selected metric for the whole project's history. The ring sections that represent packages in the Sunburst Diagram are colored with the recursive average value of its children, allowing for objective comparison between packages, and not only classes.

\begin{figure}[H]
  \centering
  \includegraphics[width=1\textwidth]{figures/colormap_hier.png}
  \caption{Treemap and Sunburst Diagram color coded with the Sequential Colormap}
  \label{fig:colormap_hier}
\end{figure}

The divergent colormap can be used to analyze the activity between revisions. On Figure \ref{fig:colormap_div_sun}, we can see that a lot of effort has been put in the current revision in order to decrease the size of most classes, with exception of the ones in package A, which have increased significantly. This might suggest that refactoring with code movement took place in this revision.

\begin{figure}[H]
  \centering
  \includegraphics[width=1.0\textwidth]{figures/colormap_div_sun.png}
  \caption{Divergent Colormap clamped at $\pm50\%$ metric change}
  \label{fig:colormap_div_sun}
\end{figure}

We can also use the color mapping technique with the treemap in order to try to find interesting pattern occurrences between the LOC metric (i.e. rectangle area) and other arbitrary metrics. On Figure \ref{fig:number_methods}, we have used the Sequential Colormap to illustrate the Number of Methods per Class metric on the last revision of the Exoplayer project. Naturally, large classes tend to have more methods than small ones, but it is also possible to notice three classes with over a hundred methods inside of each, which might indicate code that is very hard to understand and maintain.

\begin{figure}[H]
  \centering
  \includegraphics[width=\textwidth]{figures/number_methods}
  \caption{ExoPlayer Treemap with number of methods per class metric }
  \label{fig:number_methods}
\end{figure}
