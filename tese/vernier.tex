\documentclass[cic,tc,english]{iiufrgs}
% Outras Opções:
% * english    -- para textos em inglês
% * openright  -- Força início de capítulos em páginas ímpares (padrão da
% biblioteca)
% * oneside    -- Desliga frente-e-verso
% * nominatalocal -- Lê os dados da nominata do arquivo nominatalocal.def

\usepackage[utf8]{inputenc}   % pacote para acentuação
\usepackage[alf]{abntex2cite}
\usepackage{graphicx}
\usepackage{times}
\usepackage[alf,abnt-emphasize=bf]{abntex2cite}	% pacote para usar citações abnt
\usepackage [english]{babel}
\usepackage [autostyle, english = american]{csquotes}
\MakeOuterQuote{"}
\usepackage[english]{babel}
\usepackage[nottoc]{tocbibind}
\usepackage{graphicx}
\usepackage{indentfirst}
\usepackage{hyperref}
\usepackage{float}
\usepackage{wrapfig}
\usepackage{subcaption}
\usepackage{cleveref}
\usepackage{tabularx}
\usepackage{amsmath}
\usepackage{amsfonts}
\usepackage{amssymb}
\usepackage{subfiles}
\usepackage{blindtext}
%
% Informações gerais
%
\title{Visualization of the Evolution of Software Quality Metrics on Open Source Projects}

\author{Vernier}{Eduardo Faccin}

\advisor[Prof.~Dr.]{Comba}{Joao L.D.}
\coadvisor[Prof.~Dr.]{Telea}{Alexandru C.}

% a data deve ser a da defesa; se nao especificada, são gerados
% mes e ano correntes
\date{December}{2016}

% o local de realização do trabalho pode ser especificado (ex. para TCs)
% com o comando \location:
\location{Porto Alegre}{RS}

% itens individuais da nominata podem ser redefinidos com os comandos
% abaixo:
% \renewcommand{\nominataReit}{Prof\textsuperscript{a}.~Wrana Maria Panizzi}
% \renewcommand{\nominataReitname}{Reitora}
% \renewcommand{\nominataPRE}{Prof.~Jos{\'e} Carlos Ferraz Hennemann}
% \renewcommand{\nominataPREname}{Pr{\'o}-Reitor de Ensino}
% \renewcommand{\nominataPRAPG}{Prof\textsuperscript{a}.~Joc{\'e}lia Grazia}
% \renewcommand{\nominataPRAPGname}{Pr{\'o}-Reitora Adjunta de P{\'o}s-Gradua{\c{c}}{\~a}o}
% \renewcommand{\nominataDir}{Prof.~Philippe Olivier Alexandre Navaux}
% \renewcommand{\nominataDirname}{Diretor do Instituto de Inform{\'a}tica}
% \renewcommand{\nominataCoord}{Prof.~Carlos Alberto Heuser}
% \renewcommand{\nominataCoordname}{Coordenador do PPGC}
% \renewcommand{\nominataBibchefe}{Beatriz Regina Bastos Haro}
% \renewcommand{\nominataBibchefename}{Bibliotec{\'a}ria-chefe do Instituto de Inform{\'a}tica}
% \renewcommand{\nominataChefeINA}{Prof.~Jos{\'e} Valdeni de Lima}
% \renewcommand{\nominataChefeINAname}{Chefe do \deptINA}
% \renewcommand{\nominataChefeINT}{Prof.~Leila Ribeiro}
% \renewcommand{\nominataChefeINTname}{Chefe do \deptINT}

%
% palavras-chave
% iniciar todas com letras minúsculas, exceto no caso de abreviaturas
%
\keyword{Sotware quality metrics}
\keyword{Metric extraction}
\keyword{Dimensionality reduction}
\keyword{Hierarchical data}
\keyword{Evolution}

%\settowidth{\seclen}{1.10~}

%
% inicio do documento
%
\begin{document}

% folha de rosto
% às vezes é necessário redefinir algum comando logo antes de produzir
% a folha de rosto:
% \renewcommand{\coordname}{Coordenadora do Curso}
\maketitle

% dedicatoria
% \clearpage
% \begin{flushright}
%     \mbox{}\vfill
%     {\sffamily\itshape
%       ``If I have seen farther than others,\\
%       it is because I stood on the shoulders of giants.''\\}
%     --- \textsc{Sir~Isaac Newton}
% \end{flushright}

% agradecimentos
%\chapter*{Agradecimentos}
%Agradeço ao \LaTeX\ por não ter vírus de macro\ldots


% resumo na língua do documento
\begin{abstract}
  Computer scientists have done an incredible job at creating rich visualizations for fields such as engineering, chemistry, physics, and medicine. Yet, in modern software development, it is very hard to find visualization tools at use in design, implementation, maintenance, or testing of code. This work attempts to provide a set of visualization techniques that facilitate the understanding of the evolution of software entities and their relationships, supplying professionals with interesting insights about the development process and product.
\end{abstract}

% resumo na outra língua
% como parametros devem ser passados o titulo e as palavras-chave
% na outra língua, separadas por vírgulas
\begin{englishabstract}{Visualizando a evolução de métricas de qualidade de software em projetos Open-Source}{Métricas de qualidade de software. Redução de dimensionalidade. Dados hierárquicos. Evolução}
Cientistas da computação têm feito um ótimo trabalho na criação de ricas visualizações para disciplinas como engenharias, química, física e medicina. Entretanto, no processo de desenvolvimento de software moderno, é raro encontrar ferramentas de visualização em uso no design, implementação, manutenção ou teste de código. Este trabalho tenta prover um conjunto de técnicas que visam facilitar a compreensão de entidades de software e seus relacionamentos, permitindo maior discernimento do produto e processo de desenvolvimento de software.
\end{englishabstract}

% lista de figuras
\listoffigures

% lista de tabelas
\listoftables

% lista de abreviaturas e siglas
% o parametro deve ser a abreviatura mais longa
\begin{listofabbrv}{SPMD}
    \item[SCIVIS] Scientific Visualization
    \item[INFOVIS] Information Visualization
    \item[SOFTVIS] Software Visualization
    \item[MRI] Magnetic Resonance Imaging
    \item[VCS] Version Control System
    \item[GUI] Graphical User Interface
    \item[MP] Multidimensional Projections
    \item[OOP] Object-oriented Programming
\end{listofabbrv}

% idem para a lista de símbolos
% \begin{listofsymbols}{$\alpha\beta\pi\omega$}
%     \item[$\sum{\frac{a}{b}}$] Somatório do produtório
%     \item[$\alpha\beta\pi\omega$] Fator de inconstância do resultado
% \end{listofsymbols}

% sumario
\tableofcontents

% aqui comeca o texto propriamente dito

\subfile{sections/1_introduction.tex}

\subfile{sections/1.5_related.tex}

\subfile{sections/2_metric_collection.tex}

\subfile{sections/3_visualization.tex}
  \subfile{sections/3.1_hierarchical.tex}
  \subfile{sections/3.2_colormap.tex}
  \subfile{sections/3.3_projections.tex}
  \subfile{sections/3.4_glyphs.tex}
  \subfile{sections/3.5_evolution.tex}
  \subfile{sections/3.6_linking.tex}
  \subfile{sections/3.7_highlighting.tex}

\subfile{sections/4_conclusion.tex}



% referências
% aqui será usado o environment padrao `thebibliography'; porém, sugere-se
% seriamente o uso de BibTeX e do estilo abnt.bst (veja na página do
% UTUG)
%
% observe também o estilo meio estranho de alguns labels; isso é
% devido ao uso do pacote `natbib', que permite fazer citações de
% autores, ano, e diversas combinações desses

\bibliographystyle{abntex2-alf}
\bibliography{biblio}
\end{document}
